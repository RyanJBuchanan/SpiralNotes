\documentclass{article}

\usepackage{amssymb}
\usepackage{mathtools}
\usepackage{amsmath}
\usepackage{tikz}
\usepackage{tikz-cd}
\usepackage{quiver}

\title{The Anisotropy of Factions}
\author{Ryan J. Buchanan}
\date{September 8, 2023}

\newtheorem{dn}{Definition}
\newtheorem{as}{Assumption}
\newtheorem{rk}{Remark}

\begin{document}
	\maketitle
	
	\section{Prologue}
	
	Let $\mathfrak{N}$ be a network, $\Sigma_\mathfrak{A}$ a collection of actors, and let their be a predefined transport structure, $\varphi=f(\breve{a}_i\in\mathfrak{A})$ such that, for any two actors, $\breve{a},\breve{a}'$, there is some relationship $\breve{a}R\breve{a}'$ which fulfills the function f($\breve{a}_i$). We then define the relationship tautologically by writing
		$$R\coloneq \mathfrak{f}_i(x) \xrightarrow{\varphi} \mathfrak{f}_{i'}(y)$$
	where $\mathfrak{f}_i$ denotes some predicate ($\emph{fact}$), indexed by a set $\mathcal{I}\sim\mathfrak{F}$, for $\mathfrak{F}$ some faction, and where x,y correspond to transformations of $\Sigma_\varphi(\breve{a},\breve{a}')$. Here, $\Sigma$ should be taken to symbolize a $\emph{reflectivity}$ operator. 
	
	The capacities of an actor are twofold. Firstly, an actor may $\emph{express}$ some internal state of affairs via a projective, directed morphism (presumably towards some other actor in $\mathfrak{N}$). Secondly, an actor is equipped with some internal facility for $\emph{processing}$ those morphisms of which it is the target. For a map $a \xleftarrow{\varphi} b$, we may write $\Sigma_\varphi(b,a)$ for the internal reflectum generated by said map.
	
	It may very well be the case that the exact computation involved in producing the output of $\Sigma_\varphi(\ast,\ast)$ is not known to us, or perhaps is altogether undefinable. We make the following assumption:
	
	\begin{as}
		Given a map, $\varphi: S\;\to\;T$, where $S \neq T$, 
		
		$$\dot{\alpha}(S) \neq \dot{\alpha}(T)$$
	\end{as}
	
	We define the function $\dot{\alpha}(\ast) = \ast_\mathbb{R}^\dagger$, called the $\emph{idiolectic instantiation}$ of some fact $\mathfrak{f}_\xi \in \mathbb{R}$ now. 
	
	Assume that there is some ``proper truth-hood," $\mathbb{T}$. It naturally follows that:
	
	\begin{as}
		For every truth value, $\tau$, accorded to some proper fact $\mathfrak{f}$ by some agent $\dot{a}^\dagger$, either of the following cases hold. There is an item, $\dot{a}^\dagger(\mathfrak{f})$ corresponding to the assessment, which constitutes a fact in its own right. Either of the following hold:
		
		\begin{enumerate}
			\item{$\dot{a}^\dagger(\mathfrak{f})$ is a $\emph{subset}$ of $\mathbb{T}$
			\item{$\dot{a}^\dagger(\mathfrak{f})$ is a $\emph{distortion}$ of some (canonically) $\emph{true fact}$, $\mathfrak{f}\in\mathbb{T}$}}
		\end{enumerate}
		
	\end{as}
	
	These scenarios may hold simultaneously, but (at least) one must hold. In general,
	
	$$\breve{a}_n(\mathfrak{f}_\theta) \models (\mathbb{T}|_{\mathfrak{F}_\theta})|_{\breve{a}_n}$$
	
	\begin{rk}
		Note that there is a bijection 
		\begin{equation}
			(\mathbb{T}|_{\mathfrak{F}_\theta})|_{\breve{a}_n} \xleftrightarrow{\tau|_{int(\breve{a})}=1} \tau(\dot{\alpha}(\mathfrak{f})|_{t=0})
		\end{equation}
		
	which means that there is a correspondence between the $\emph{internal}$ conception of a fact by an actor $\breve{a}_n$, and the actor's $\emph{assessment}$ of the correspondence of the fact with the ``reality" of things (as constructed by a faction, $\mathfrak{F}$) at some arbitrary time t=0.
	\end{rk}
	
	Assumption 1 may then be interpreted in plain English by saying that the bijection laid out in Remark 1 varies as with each agent. Thus, in communicating some concept, $\sigma$, to an agent x, the output is not $\sigma$ itself, but some transformed variation of $\sigma$, say, $x(\sigma)$. This transformation occurs at the level of signification; i.e., the signified remains constant; and yet, it is not the signified which is interpreted, but some internal reflection of the signifier. Thus, if I wish to communicate a concept to you, it is impossible for me to express myself in a manner such that your impression corresponds to an identical fact ``in the real world" as mine.
	
	\begin{dn}
		An idiolectic instantiation, $\bar{i}$, is a local bijection, $\varphi_\star$ between the semantic field of an agent and some faction $\mathfrak{F}$ lying within the domain of absolute truth-hood such that there is a faction $\mathfrak{F}' = \mathfrak{F}|_{\breve{a}}$ whose constitution enables
		
		$$\tau(\varphi_\star) = 1$$
		
		to hold.
	\end{dn}
	
	Where we allow 1. and 2. of (Assumption 2) to hold simultaneously, we permit that absolute truth is, in some sense, arbitrary, such that for every proper fact $\mathfrak{f}\in\mathbb{T}$, every distortion $d(\mathfrak{f})$ also lies in $\mathbb{T}$. In such a case, ``absolute truth" is an incredibly refined (to abuse the term) $\emph{moduli space}$ which all relative, idiolectic instantiations are confined to. Thus,
	
	\begin{equation}
		\forall \vartheta(\mathfrak{f})\in\mathbb{T}|_\mathfrak{F} \;\; \exists^\theta \mathfrak{f}' \in \mathbb{T}
	\end{equation}
	
follows. Here, $\exists^\theta$ denotes an existential quantifier whose membership condition is $\theta$. Letting $\theta$ denote orientability, we obtain that for every fact belonging to the restriction of absolute truth to an actor-network (faction), there is some rotation of an alternate (absolute) fact which corresponds to the idiolectic instantiation of $\mathfrak{f}$ by $\vartheta$.

% https://q.uiver.app/#q=WzAsMTAsWzAsMSwiXFxtYXRoZnJha3tmfSBcXGluIFxcbWF0aGJie1R9Il0sWzIsMSwiXFxtYXRoYmJ7Un18X1xcbWF0aGZyYWt7Rn0iXSxbMywwLCJcXGFscGhhIl0sWzMsMSwiXFxiZXRhIl0sWzMsMiwiXFxnYW1tYSJdLFswLDQsIlxcbWF0aGZyYWt7Zn0nIFxcaW4gXFxtYXRoYmJ7VH0iXSxbMiw0LCJcXG1hdGhiYntSfV97XFxtYXRoZnJha3tGfSd9Il0sWzMsMywiXFxhbHBoYSJdLFszLDQsIlxcYmV0YSJdLFszLDUsIlxcZ2FtbWEiXSxbMCwxLCJcXGRhZ2dlciJdLFsxLDIsIngiXSxbMSwzLCJ5IiwxXSxbMSw0LCJ6IiwyXSxbNSw2LCJcXGRhZ2dlciJdLFs2LDcsInoiXSxbNiw4LCJ4IiwxXSxbNiw5LCJ5IiwyXSxbMCw1LCJcXHRoZXRhIiwxLHsic2hvcnRlbiI6eyJzb3VyY2UiOjIwLCJ0YXJnZXQiOjMwfSwic3R5bGUiOnsidGFpbCI6eyJuYW1lIjoiYXJyb3doZWFkIn19fV0sWzIsMywiXFx2YXJwaGkiLDAseyJjdXJ2ZSI6LTEsInN0eWxlIjp7InRhaWwiOnsibmFtZSI6ImFycm93aGVhZCJ9fX1dLFszLDQsIlxcdmFycGhpIiwwLHsiY3VydmUiOi0xfV0sWzcsOCwiXFx2YXJwaGkiLDAseyJjdXJ2ZSI6LTEsInN0eWxlIjp7InRhaWwiOnsibmFtZSI6ImFycm93aGVhZCJ9fX1dLFs4LDksIlxcdmFycGhpIiwwLHsiY3VydmUiOi0xLCJzdHlsZSI6eyJ0YWlsIjp7Im5hbWUiOiJhcnJvd2hlYWQifX19XV0=
\[\begin{tikzcd}
	&&& \alpha \\
	{\mathfrak{f} \in \mathbb{T}} && {\mathbb{R}|_\mathfrak{F}} & \beta \\
	&&& \gamma \\
	&&& \alpha \\
	{\mathfrak{f}' \in \mathbb{T}} && {\mathbb{R}_{\mathfrak{F}'}} & \beta \\
	&&& \gamma
	\arrow["\dagger", from=2-1, to=2-3]
	\arrow["x", from=2-3, to=1-4]
	\arrow["y"{description}, from=2-3, to=2-4]
	\arrow["z"', from=2-3, to=3-4]
	\arrow["\dagger", from=5-1, to=5-3]
	\arrow["z", from=5-3, to=4-4]
	\arrow["x"{description}, from=5-3, to=5-4]
	\arrow["y"', from=5-3, to=6-4]
	\arrow["\theta"{description}, shorten <=11pt, shorten >=16pt, tail reversed, from=2-1, to=5-1]
	\arrow["\varphi", curve={height=-6pt}, tail reversed, from=1-4, to=2-4]
	\arrow["\varphi", curve={height=-6pt}, from=2-4, to=3-4]
	\arrow["\varphi", curve={height=-6pt}, tail reversed, from=4-4, to=5-4]
	\arrow["\varphi", curve={height=-6pt}, tail reversed, from=5-4, to=6-4]
\end{tikzcd}\]

In the above diagram, some unknowable, absolute truth is projected to factions $\{\mathfrak{F},\mathfrak{F}'\}$ and then to actors $\vartheta=\{\alpha,\beta,\gamma\}$. Note that, in each faction, a different arrow corresponds to idiolectic realization of the fact by each actor. For instance, in the map 
	$$\mathfrak{F} \to \mathfrak{F}'$$
we obtain
	$$(\mathbb{R}_\mathfrak{F} \xrightarrow{x} \alpha) \Rightarrow (\mathbb{R}_\mathfrak{F'} \xrightarrow{z} \alpha) $$
so that a different set of circumstances is required to interpret two distinct facts as identical. 
\end{document}