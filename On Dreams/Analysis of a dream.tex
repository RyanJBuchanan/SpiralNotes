\documentclass{article}

\title{Analysis of a Dream}
\author{Ryan J. Buchanan}
\date{September 1, 2023}

\newtheorem{dn}{Definition}

\usepackage{amsfonts}
\usepackage{mathtools}
\usepackage{amsmath}
\usepackage{amssymb}

\begin{document}
	\maketitle
	
	\begin{abstract}
		This is a brief analysis of a dream session I had last night. I relate the themes of the dream to symbols including the objet a, and at the end introduce some notation.
	\end{abstract}
	
	\section{Prelude}
	I will begin with a sample from a dream I had last night.
	\subsection{Sample}
		I was in a rocket ship with three other people, of whom I do not remember many details about, although one of them resembled my high-school friend Gavin Severson. I exited through a small doggie-door styled flap immediately before launch, in effect ``chickening out." For a very brief period, I experienced a third-person camera which followed the rocket until it left Earth's atmosphere. Then, the camera cut to my location, where I stood in a small corridor. Apparently, the rocket had been indoors; there was a convertible car styled door at the top, which was comprised of eight teeth that locked together in the shape of a spiral.

		I spoke with a fairly thin blonde woman, who adorned what appeared to be an amazonite ring on her finger. She asked why I didn't go, and I lamented the fact out loud to her. She informed me that there would be another rocket landing in the next few hours, and I was welcome to stay for dinner.
		
		I don't remember what happened next, but I do recall a large feast laid out on a table, despite the fact that only a few people attended. I ate turkey and a string cheese. I asked how often the rockets return. A black man with a blue shirt and a boisterous voice let me know they were like a bus, and they came every so often, but told me to make sure I still had my rocket pass. 
		
		Some time later, I was back on the rocket, and this time there was a different cast of characters surrounding me. For some reason I missed the old crew. 
		
		I asked which part of the moon we were landing on, and someone said ``Fort Worth." 
		
		When we got to Fort Worth, we landed at a rocket stop, and there was a Taxi waiting for me. The car was rather weightless, and sort of glided across the surface of the moon. The taxi took me past an eight-story Burger King, at which point I commented on its size; then, we reached a building which was so tall, I could not see the top. A giant magnet attached to a chord reached down and the car became attached to it. The car was tilted at about a 60 degree angle, and I almost fell out the back window.
		
		The chord retracted upwards, and the car ascended. When we reached the twenty-eighth floor, I entered the building through the window and found a large McDonald's inside, complete with an adult-sized playground that seemed to stretch upwards and downwards endlessly. 
		
		I explored the playground for quite some time, traversing slides and platforms, and looking out little windows that were installed in random places. At some point I remember feeling overwhelmed, and amazingly lost while I was looking for my little sister, who had been younger in the dream than in real life.
		
		When I exited the playground, I found a casino that smelled like cigars, with a roulette table and an Applebee's. Jeff Bezos and a few other billionaires sat in the corner, and Bezos recognized me, and called me over to sit next to them. He told me I could have his yacht, and gave me advice to invest the thousand dollars I had.

\section{Analysis}
I went to sleep at approximately 9:30 PM last night, and awoke at around 7:00 AM. I had several small dream segments (which were rather fluid) during this time, mostly involving outer space. The dream itself, as reported, occurred after deciding to go back to sleep for as long as possible in an attempt to return to the dreams to observe their content. I slept for another four hours between 7:00 AM and 11:00 AM, intermittently waking and flopping around, with a growing urge to urinate throughout that would eventually prohibit me from dreaming further. When I woke up, I had assumed it was closer to 3:00 PM, as I usually don't go to sleep so early. 

I have a tendency to forget the main details of conversations I had in my dreams, but I remember the topics most typically. Further, I do not remember many of the characters; they are like fillers. Perhaps this is a tactic my ego uses to maintain its identity during sleep, so that it may be transferred back to myself safely without any confusion as to who I am. By conserving energy on memorizing the appearances (or names) of others, I am able to focus on my own world. I believe this technique is adopted sometime in early childhood, after or during the mirror phase, when a child begins not to identify itself with its surrounding world.

The tendency to view my dreams in the third person represents an acute consciousness of my appearance to others. This may be a ``training strategy" my mind employs to foster the perception of empathy. This is to say, by acting in a way I believe others will view as the ``right" way, I will appear as a more empathetic person, which is something that was heavily ingrained in me in my early childhood therapy as an Autistic individual. It can also represent the struggle to be a mature, responsible adult, and a desire to adopt a more passive lifestyle where I can, in effect, watch my life as a movie without being responsible for my actions.

The billionaires towards the end of my dream represent the striving of my superego to obtain higher status. This is common, although particular to my case is the past experience of homelessness and persistent poverty, which may further the drive to interact with or be categorized alongside such people. Notably, the blonde woman (perhaps inspired by the movie ``Barbie") adorned what I believed to be an amazonite ring. Bezos owns Amazon; Elon Musk owns spacex, which may have been the company that transported me to the moon. To further comment on the blonde lady, she may have been even a maternal figure, and the dialogue we had resembles the conversations I had with my real mother when I failed college and dropped out of Job Corps.

The moon represents the Jungian shadow ego. Here, I engage in activities that are considered taboo: playing on a giant playground, entering a casino, etc. The playground symbolizes the trauma I experienced as an adolescent, when I was ``thrown in" (to use Heidegger's terminology) to a world where I was no longer a child, and thus was forced to seek satisfaction outside the maternal symbol. This is indicative of an Oedipus complex; playgrounds, cartoons, and other comforts of the childhood come to stand in for the mother, who herself stands in for the breast, as these are the socially denied comforts of the adolescent peer group. This is the second denial; the first denial is the denial of the real breast, once the child is weened; the second is the denial of the symbolic breast. Through denial, one develops their relationship with the objet a. The relative anonymity, or facelessness of my dream characters may possibly be a manifestation of the social rejection I experienced in middle school, and the coldness of space represents my withdrawal into the Autistic world.

\begin{equation}
	\mathbb{R}_{den} \xrightarrow{\hat{a}} Peers
\end{equation}

The magnet and chord which rose the taxi to the twenty-eigth story resembled a contraption from a junkyard; in particular, there is a scene in the childhood movie ``The Brave Little Toaster" which may very well have inspired this vision. Applebee's symbolizes my older cousin Tyann Applebee, whom I showered with when I was very little. The fact that I ate nothing at Applebee's is indicative of the social impossibility of ``indulging" in a phallic fantasy with her, even on the moon, and even in the shadow.

There are a slough of reasons why Forth Worth was a city on the moon in my dream. My sister Kiya, and two of my friends, Keaten and Melissa, visited or moved to Texas in the last year or so. ``Worth" is very similar to the word ``value," and my thinking about truth has led me to contemplate ``truth values" quite a bit recently. A fort may also be a reference to the childhood blanket forts that Tyann, my sister Ashley and I would build when I was a wee lad.

\subsection{Symbolism}
The following equation represents the condensation of small objects in the actual world to the fusion-state of the dream:
\begin{equation}
	\sum_{i=0}^{\mathcal{N}} \delta_i(\gamma) \mapsto \mathfrak{D}
\end{equation}

and the following represents the construction of a real mood with which one awakens:

\begin{equation}
	\sum_{i=0}^{\infty} \delta_i(\mathfrak{d}) \mapsto \mathfrak{R}_{mood}
\end{equation}

By $\infty$, we mean an $\emph{infinity ideal}$; i.e., a perfectly smooth interval which is the least upper bound for the ring of numbers at hand. The dream state is obtained by summing over the effective components of one's actions, $\gamma$, along with their reflections, $\Sigma_\gamma(x)$. Conversely, it is the affective states of the dream which charge one's waking mood. 

The domain of potentials differs for each equation; for (eq. 2), it is $\mathcal{U}^\infty$, and for (eq. 3), it is $\mathfrak{U}|_\mathbb{R}$. While the facts of the real universe remain constant over time, their internal reflections change. So, as n changes:

\begin{equation}
	\underset{n\to\mathcal{N}}{t_n} = \underset{x \to x'}{\Sigma_\gamma(x)}
\end{equation}

and as a result, the correspondences of statements change in their relationship to reality. Thus, the facts of the matter are preserved, and yet, the $\emph{factions}$ they form, undergo evolution in a long exact sequence. One may wonder how to model this scenario using loop stacks, or infinite loop spaces.
\end{document}
